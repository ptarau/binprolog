\documentclass{article}
\usepackage{url}

\textheight 21 cm
\textwidth 14 cm
 
\topmargin -1mm
\marginparwidth 0mm
\evensidemargin 1.5 cm
\oddsidemargin 1.5 cm

\begin{document}
\pagestyle{plain}
\bibliographystyle{abbrv}

\title{
\Huge
BinProlog 2006 version 11.x Professional Edition\\
User Guide\\
\vskip 5cm
}

\author{
\Large
  {\bf Paul Tarau}\\\\
\large
  BinNet Corp.\\
  e-mail: binnetcorp@binnetcorp.com\\
  WWW: http://www.binnetcorp.com
\normalsize
}

\date{}

\maketitle

\vskip 5cm

\newpage

\section{Installation}

{\em
Copy the BinProlog executable for your architecture to 
something called {\bf bp.exe} or {\bf bp} somewhere in your path, then type `bp'.
BinProlog standalone executables are usually provided in subderectory {\bf bin} of
the distribution.}

For {\em multi-user} installations set the {\tt BP\_PATH} variable 
to point to the BinProlog source directory, for instance {\bf /bp\_dist/src}. 
This will allow users to load
example programs in directory {\bf /bp\_dist/progs} without providing path 
information  and will also make include commands like :-[library(lists),
refering to {\bf /bp\_dist/library} accessible from any directory.

Normally the appropriate {\bf bp.exe} or {\bf bp} file (a self contained
executable) is all you need to have BinProlog  running.  For
PCs, just copy the bp.exe Windows executable file somewhere in your PATH.  

To start BinProlog use

\begin{verbatim}
$ bp <command-line options> <wam-bytecode-file> or <prolog-file>
\end{verbatim}

or simply

\begin{verbatim}
$ bp
\end{verbatim}

Sizes of the blackboard, heap, stack, trail and code areas can be passed as
command line parameters etc., as well as other options, as shown
with {\tt bp -x} help request:

\begin{verbatim}
With bp -x you can list the following:

COMAND LINE OPTIONS:

-h ==> HEAP SIZE in Kbytes,  default:1024K
-s ==> STACK SIZE in Kbytes, default:256K
-t ==> TRAIL SIZE in Kbytes, default:256K
-c ==> CODE SIZE in Kbytes,  default:1024K
-b ==> BLACKBOARD SIZE in Kbytes, dynamic if 0, default:512K
-a ==> MAX. ATOMS, give exponent of 2, 2^<value>, default=2^16=65536
-d ==> HASH DICT. entries, give exponent of 2,  2^<value>, default 2^16=65536
-i ==> IOBUFFER, in bytes, default:131072bytes
-q ==> QUIETNESS level, default:2 (lower means more verbose)
-l ==> LOAD_METHOD:
  (1=mcompile, 2=scompile, 3=oconsult, 4=dconsult, 5=sconsult): 1
-r ==> call/update RATIO controlling dynamic recompilation:10
-p ==> PORT to run on as a daemon if >0, 1 shorthand for default_port, current d
efault:0, -p10 forces detection of current IP address

ARGUMENTS:
  STARTUP FILE: *.bp, *.pl, *.wam or
  GOAL: pred(args),
  default: wam.bp


Prolog execution halted(2). CPU time = 0.241s

ARGUMENTS:
  STARTUP FILE: *.bp, *.pl, *.wam or
  GOAL: pred(args),
  default: wam.bp
\end{verbatim}

{\flushleft Type}

\begin{verbatim}
?- help(<word>).
\end{verbatim}

{\flushleft and} then use {\tt info/1} with the matching predicate names
to get a short description and possibly an example of use:

\begin{verbatim}
?- info(name/arity).
\end{verbatim}

{\flushleft Type}

\begin{verbatim}
?-info.
\end{verbatim}

{\flushleft for} a (long) description of builtins and examples.

\section{Deprecated Predicates}

BinProlog  has undergone a number of simplifications and as a result, older,
seldomly used functionality has been removed. Please take a look at the file
{\bf library/deprecated.pl} for the code of the removed predicates. By including
the file in your older programs, most of the functionality of the removed
predicates can be reactivated. The changes include replacement of the Assumption Grammar
operations (now part of standard DCG processing) and mobile code+client/server 
networking (now replaced by robust Jinni  compatible code which
supports socket reuse for improved performance - see help(rpc)).

\section{Obtaining BinProlog}

The ORIGINAL DISTRIBUTION SITE for BinProlog\footnote{BinProlog Copyright $\copyright$ Paul Tarau 1992-98 and BinNet Corp. 1998-2006. All rights reserved} is:

\begin{verbatim}
   http://www.binnetcorp.com/BinProlog
\end{verbatim}

Please send comments and bug reports to {\em binnetcorp@binnetcorp.com .}

\section{Release Notes for This Version of BinProlog}

{\small \begin{verbatim}
Welcome to BinProlog Full Source Edition and BinProlog Professional Edition!

!!! after you work with them on Windows some makefiles might need conversion 
    to remove ^M character on Unix systems

!!! Please look for *.bat files which usually automate frequent maintenance 
    operations on XP, 2000, NT or Win9x PCs.

With Microsoft VCC based precompiled distributions for Windows XP/2000/NT/98/95,

-a ready to run binary (bp.exe) is included in directory bin,
 as well as a redistributable runtime bpr.exe for Windows  

-a packaging of BinProlog as a DLL and examples are available in 
 directory BP_DLL.

-header files in static libraries are available in directory lib - allowing
 to use the C-interface or generate C-code without need to recompile the sources

Just copy the executables bp.exe or bpr.exe somewhere on your path, 
and bp_lib.dll and bpr_lib.dll to the Windows or Winnt/System32 directory, 
or simply copy them to the local directory where you keep your application.

BinProlog's C-interface tools are in directory c_inter.

Tools for generation of standalone executables, through compilation to C
are available in directory pl2c.

BinProlog's Tcl/Tk interface is in directory TCL. Is ready to run, on
all platforms (new client/server design), no special make actions are 
needed.

Finally, the directory doc, in distributions coming with full documentation,
contains the documentation in PostScript and HTML form - note the new API 
description in file help.html, also generated whin just typing "help" in 
BinProlog.

--------------------------------------------------------------------------
In case you have BinProlog Full Source Edition, the directory src
contains makefiles for gcc, VCC 6.0 and .NET C/C++ batch files to rebuild the
sources after you make changes. Just go in directory src and type
make all or makeall.bat

Please read the documentation files provided separately and visit
our ONLINE HELP SYSTEM at BinNet Corp.'s web site, for last minute
updates, demos and information about new features and components.

On Cygnus' CYGWIN gcc, taking

  make all

which also works on normal Unix systems not requiring unusual flags.

On Linux PCs type

  make linux_mt
  
  or
  
  make linux

On Solaris sparcs type

  make solaris

This will take care to generate the content of most of the directories.

Finally, the Full Source Edition also contains VCC 6.0 project files
in deirectory winbp, for working conveniently with the sources on a 
Windows NT workstation.

Alternatively, work on sources is now also supported for Cygnus gcc
as well as Linux and othe Unix platforms, through a complete set of
updated makefiles.

----------------------------------------------------------------------
Additional components, available on some platforms.

The directory csocks contains tools for building standalone C-based
socket based client, server and a remote toplevel components 
based on BinProlog's modular and portable socket package.

Tools to build an ISO parser for BinProlog are in directory cparser.
\end{verbatim}}

\section{Using BinProlog}

\subsection{Learning Prolog}

If you are just starting, you might want to take a look at one of
the following online tutorials. Most of their content applies to
any Prolog around, and in particular to BinProlog which a fairly
standard Prolog system.

\begin{verbatim}
Online Prolog Introductions/Tutorials:

  http://kti.ms.mff.cuni.cz/~bartak/prolog/learning.html 
  http://www.cs.bham.ac.uk/~pjh/prolog_course/sem223.html 
  http://www.cs.sfu.ca/CC/SW/Prolog/Notes/toc.html 
\end{verbatim}

The latest version of BinProlog is usable directly over the
Internet (follow the {\em demo} and {\em query} links from \url{http://www.binnetcorp.com}).

\subsection{Consulting/compiling files}

BinProlog features a number of different compilation and consulting
methods as well as dynamic recompilation of consulted (interpreted) code
for fast execution.

\noindent The shorthand

\begin{verbatim}
   ?- [myFile].
\end{verbatim}

\noindent defaults to the last used compilation method applied to
{\tt myFile.pro} {\tt myGile.pl} or {\tt myFile}. Among them,
the default mcompile/1 comiling to memory and scompile/1 which
uses temporary *.wam files and to quicly load files which have not been
changed. A good way to work with BinProlog is to make a "project" *.pro
file includeing its components, as in:

\begin{verbatim}
:-[myFile1].
:-[myFile2].
..............
\end{verbatim}

\noindent  The shorthand for {\tt co(myFile)},

\begin{verbatim}
    ?- ~myFile.
\end{verbatim}

\noindent defaults to the last used {\em interpretation} method applied to
{\tt myFile.pro} {\tt myFile.pl} or {\tt myFile}.
For online information
on their precise behavior, do:

\begin{verbatim}
?-info(oconsult),info(consult),info(dconsult),info(sconsult).

oconsult/1: 
  reconsult variant, consults and overwrites old clauses

consult/1: 
  consults with possible duplication of clauses, allows 
  later dynamic recompilation

consult/2: 
  consult(File,DB) consults File into DB

dconsult/1: reconsult/1 variant, cleans up data areas, 
  consults, allowing dynamic recompilation

sconsult/1: 
  reconsult/1 variant: 
    cleans up data areas, consults, makes all static
\end{verbatim}

To control dynamic recompilation you can use {\tt dynco/1} with {\tt yes}
and {\tt no} as arguments or {\tt db\_ratio/1} to precisely specify
the ratio between calls and updateds to predicates which will
trigger moving it from interpreted to compiled representation
(default=10).

{\em BinProlog's compilation mode is intended to work with ONE toplevel
project file including various components.} For interactive development
use: 

\begin{verbatim}
~file1.
~file2.
......
~fileN.
\end{verbatim}

\noindent Note that, by default, older definitions from previous or the same file
of predicates with same name/arity are quietly overwritten by default.

\begin{quote}
Do not worry about performance. List your predicates at will, using listing/1 or debug them with trace/1, modify them with assert/retract. 
Automagically, BinProlog will
override heavily used predicates with compiled variants. This usually
pushes performance to about half of optimized compiled code, while
giving you the full flexibility of interpreted code.
\end{quote}

\subsection{The interactive toplevel shell}

To see the command line options:

\begin{verbatim}
   $ bp -x
\end{verbatim}

{\flushleft To} compile and load a file possibly having one of
the sufixes `pl' or `pro', do:

\begin{verbatim}
   ?-[<file>].
\end{verbatim}

{\flushleft To} quickly consult a file for interpreted execution, having one of
the sufixes `pl' or `pro', do:

\begin{verbatim}
   ?-co(<file>).
\end{verbatim}

\paragraph{Search path mechanism.} BinProlog
searches for programs in the directories {\tt ., ./progs}
 and {\tt ../src ../library}, relative to
the {\tt BP\_PATH} environment variable which defaults
silently to the current directory if undefined.
The directory {\tt ./myprogs} is also searched for possible
user programs. While compatible to previous configurations
which where single user minded, this allows installing the
BinProlog distributon for shared use in a place like 
{\tt /usr/local/BinProlog}\footnote{This is usually better to
be left to a system person who 	also can ensure that users
inherit the {\tt BP\_PATH} environment variable. An individual user
can also put something like {\em setenv BP\_PATH /usr/local/BinProlog}
his or her {\tt .cshrc} or {\tt .login} file, to access the shared
BinProlog programs.} In addition, the environment variable {\tt PROLOG\_PATH}
provides a secondary search directory for including an 
individual user's Prolog library files.

{\flushleft Mode}
is interactive by default (for compatibility with other Prologs) but if
you use a modern, windows based environment you may want to switch it
off with:

\begin{verbatim}
?- interactive(no).
\end{verbatim}

{\flushleft or} turn it on again with

\begin{verbatim}
?- interactive(yes).
\end{verbatim}



{\flushleft Operators} are defined and retrieved with

\begin{verbatim}
:-op/3, current_op/3.

as in:

?-op(333,xfx,and).

allowng to enter:

?-write(joe and mary),nl.

instead of the usual ?-write(and(joe,mary)).
\end{verbatim}

\subsection{Source code loading methods}

The preferred use of BinProlog is through reloading a unique
project file containing included files\footnote{This overcomes the limitation of previous versions of
having only one top-level file.}.

A smart compile facility (scompile/1) implement a basic
{\em make} facility: if the {\tt *.wam} version of the file
is newer it will be reloded very quickly instead of recompiling the
corresponding {\tt *.pl} file.

\paragraph{Including files.} 
For both compiled and consulted files use a generic include directive:

\begin{verbatim}
:-[file].
\end{verbatim}

which automatically adjusts to your current {\em load method}.
You can set your prefered loading method directly from the command line
using option {\tt \-l}. Explicite use of {\tt compile/1} or {\tt reconsult/1} can be used to restrict your selection to a current
compile method or consult method.

Use {\tt co/0} and {\tt ed/0} as a shorthand to reload/re-edit
the last compiled/consulted file.

Use {\tt \~~ file} or {\tt co(file)} as a short hand to reconsult a file with your
current consult method. This is useful if your load method is
by default a version of compile but you want to reconsult a file
so that you can debug/list clauses of verious predicates.

{\em Normally, after each compile BinProlog
cleanes up ALL its data areas. Not relaying on hidden state
of the database as most Prologs do is a very IMPORTANT
element of reliable software developpment. In compiled BinProlog
recompiling means a fresh start, so that the future of
computation is only dependent on your code, not on
what happens to be accidentally in your database.
This FEATURE is unlikely to be changed in the future.
}.

With full flexibility of interpreted code and its automatic
on the fly compilation, this limitation does not affect by
any means a user who prefers conventional Prolog semantics,
which is the default for the consulting facility co/1 or its shorthand \~~ .
There's however an ability to {\em push} your code into
the BinProlog protected kernel at runtime by typing {\tt pc}.
This allows doing another {\tt compile/1} without discarding the currently
compiled file, while protecting the first one from
accidental overriding of predicates.
A better way to proceed with multiple files is to create a {\tt *.pro}
project file and include in it all the files needed for the project
using the {\tt :-[myFile]} directive. BinProlog's built-in {\tt make}
facility (using {\tt scompile/1}) ensures quick compilation of
a project by only recompiling files that have been edited.

\section{Editing, help/1, info/1, apropos/1, trace/1, spy/1, nospy/1}

To edit a file and then compile it use:

\begin{verbatim}
   ?- edit(<editor>,<file>).
\end{verbatim}

To edit and recompile the currently compiled file using the
{\tt emacs} editor, attached to environment variables EDITOR or VISUAL type:

\begin{verbatim}
   ?- ed.
\end{verbatim}

To edit and recompile the currently compiled file using the
{\tt edit} editor (under DOS) type:

\begin{verbatim}
   ?- edit.
\end{verbatim}

To simply recompile the last file type:

\begin{verbatim}
  ?- co.
\end{verbatim}

{\flushleft The debugger/tracer} uses R.A. O'Keefe's public domain
meta-interpreter.  You can modify it in the file "extra.pl".
{\flushleft DCG-expansion} is supported by the public domain file dcg.pl.

To debug a file type:

\begin{verbatim}
?- reconsult(FileName).
\end{verbatim}

{\flushleft and then}

\begin{verbatim}
?- trace(Goal).
\end{verbatim}

{\flushleft For}
 interactivity, both the toplevel and the debugger depend on

\begin{verbatim}
?-interactive(yes).
\end{verbatim}

{\flushleft or}

\begin{verbatim}
?-interactive(no).
\end{verbatim}

{\flushleft My personal preference is using}
{\tt interactive(no)} within a scrollable window.
However, as traditionally all Prologs hassle the user after
each answer BinProlog  will do the same by default.

{\em If you forget the name of some builtin, {\tt apropos/1} (or {\tt help/1}) will
give you some (flexible up to one misspelled or missing
letter) matches with their arities, while info/1 will give you online
information on builtins.
}

BinProlog  allows debugging of dynamically recompiled code.
Load your file with reconsult or one of its variants (dconsult, oconsult).
You can benefit from efficient execution of code you do not want to
look into which gets compiled on the fly during the debugging session,
while being able to use trace/1 and listing/1 to see your predicates.

BinProlog  also features some new tracing options, i.e:

\begin{verbatim}
ENTER ==> call without tracing
l ==> listing
q,a ==> abort
p ==> toplevel Prolog query
t ==> succeed, but do not call this goal
f ==> fail, and do not call this goal
k ==> keep goal for further inspection
s ==> show saved goals instances
h ==> help
; ==> continue (default)
\end{verbatim}

The following terminal session shows an example of debugging session:
{\small \begin{verbatim}
?-reconsult(allperms).
consulting(../progs/allperms.pl)
consulted(../progs/allperms.pl)
time(consulting = 50,quick_compiling = 0,static_space = 0)
yes
?- interactive(no).
yes
?- trace(g0(3)).
Call: g0(3)
 !!! clause: g0/1
 Call: nats(1,3,_x2770)
  !!! clause: nats/3
  Call: 1 < 3
   !!! compiled((<)/2)
  Exit: 1 < 3
  Call: _x3157 is 1+1
   !!! compiled((is)/2)
  Exit: 2 is 1+1
  Call: nats(2,3,_x3159)
   !!! clause: nats/3
   Call: 2 < 3
    !!! compiled((<)/2)
   Exit: 2 < 3
   Call: _x4115 is 2+1
    !!! compiled((is)/2)
   Exit: 3 is 2+1
   Call: nats(3,3,_x4117)
    !!! clause: nats/3
    CUT
   Exit: nats(3,3,[3])
  Exit: nats(2,3,[2,3])
 Exit: nats(1,3,[1,2,3])
 Call: perm([1,2,3],_x2773)
  !!! clause: perm/2
  Call: perm([2,3],_x5527)
   !!! clause: perm/2
   Call: perm([3],_x5905)
    !!! clause: perm/2
    Call: perm([],_x6283)
     !!! clause: perm/2
    Exit: perm([],[])
    Call: insert(3,[],_x5905)
     !!! clause: insert/3
    Exit: insert(3,[],[3])
   Exit: perm([3],[3])
   Call: insert(2,[3],_x5527)
    !!! clause: insert/3
   Exit: insert(2,[3],[2,3])
  Exit: perm([2,3],[2,3])
  Call: insert(1,[2,3],_x2773)
   !!! clause: insert/3
  Exit: insert(1,[2,3],[1,2,3])
 Exit: perm([1,2,3],[1,2,3])
 Call: fail
  !!! compiled(fail/0)
 Fail: fail
 Redo: perm([1,2,3],[1,2,3])
  Redo: insert(1,[2,3],[1,2,3])
   Call: insert(1,[3],_x7793)
    !!! clause: insert/3
   Exit: insert(1,[3],[1,3])
  Exit: insert(1,[2,3],[2,1,3])
 Exit: perm([1,2,3],[2,1,3])
 Call: fail
  !!! compiled(fail/0)
 Fail: fail
 Redo: perm([1,2,3],[2,1,3])
  Redo: insert(1,[2,3],[2,1,3])
   Redo: insert(1,[3],[1,3])
    Call: insert(1,[],_x8180)
     !!! clause: insert/3
    Exit: insert(1,[],[1])
   Exit: insert(1,[3],[3,1])
  Exit: insert(1,[2,3],[2,3,1])
 Exit: perm([1,2,3],[2,3,1])
 Call: fail
  !!! compiled(fail/0)
 Fail: fail
 Redo: perm([1,2,3],[2,3,1])
  Redo: insert(1,[2,3],[2,3,1])
   Redo: insert(1,[3],[3,1])
    Redo: insert(1,[],[1])
    Fail: insert(1,[],_x8180)
   Fail: insert(1,[3],_x7793)
  Fail: insert(1,[2,3],_x2773)
  Redo: perm([2,3],[2,3])
   Redo: insert(2,[3],[2,3])
    Call: insert(2,[],_x7404)
     !!! clause: insert/3
    Exit: insert(2,[],[2])
   Exit: insert(2,[3],[3,2])
  Exit: perm([2,3],[3,2])
  Call: insert(1,[3,2],_x2773)
   !!! clause: insert/3
  Exit: insert(1,[3,2],[1,3,2])
 Exit: perm([1,2,3],[1,3,2])
 Call: fail
  !!! compiled(fail/0)
 Fail: fail
 Redo: perm([1,2,3],[1,3,2])
  Redo: insert(1,[3,2],[1,3,2])
   Call: insert(1,[2],_x8155)
    !!! clause: insert/3
   Exit: insert(1,[2],[1,2])
  Exit: insert(1,[3,2],[3,1,2])
 Exit: perm([1,2,3],[3,1,2])
 Call: fail
  !!! compiled(fail/0)
 Fail: fail
 Redo: perm([1,2,3],[3,1,2])
  Redo: insert(1,[3,2],[3,1,2])
   Redo: insert(1,[2],[1,2])
    Call: insert(1,[],_x8542)
     !!! clause: insert/3
    Exit: insert(1,[],[1])
   Exit: insert(1,[2],[2,1])
  Exit: insert(1,[3,2],[3,2,1])
 Exit: perm([1,2,3],[3,2,1])
 Call: fail
  !!! compiled(fail/0)
 Fail: fail
 Redo: perm([1,2,3],[3,2,1])
  Redo: insert(1,[3,2],[3,2,1])
   Redo: insert(1,[2],[2,1])
    Redo: insert(1,[],[1])
    Fail: insert(1,[],_x8542)
   Fail: insert(1,[2],_x8155)
  Fail: insert(1,[3,2],_x2773)
  Redo: perm([2,3],[3,2])
   Redo: insert(2,[3],[3,2])
    Redo: insert(2,[],[2])
    Fail: insert(2,[],_x7404)
   Fail: insert(2,[3],_x5527)
   Redo: perm([3],[3])
    Redo: insert(3,[],[3])
    Fail: insert(3,[],_x5905)
    Redo: perm([],[])
    Fail: perm([],_x6283)
   Fail: perm([3],_x5905)
  Fail: perm([2,3],_x5527)
 Fail: perm([1,2,3],_x2773)
 Redo: nats(1,3,[1,2,3])
  Redo: nats(2,3,[2,3])
   Redo: nats(3,3,[3])
   Fail: nats(3,3,_x4117)
   Redo: 3 is 2+1
   Fail: _x4115 is 2+1
   Redo: 2 < 3
   Fail: 2 < 3
  Fail: nats(2,3,_x3159)
  Redo: 2 is 1+1
  Fail: _x3157 is 1+1
  Redo: 1 < 3
  Fail: 1 < 3
 Fail: nats(1,3,_x2770)
Exit: g0(3)

?-  interactive(yes).

?-  trace(insert(1,[2,3],Res)).
Call: insert(1,[2,3],_x2407) <ENTER=call, ;=trace, h=help>: l

% dynamic: insert/3:
insert(A,B,[A|B]).
insert(B,[A|C],[A|D]) :- 
        insert(B,C,D).

!!! clause: insert/3
Exit: insert(1,[2,3],[1,2,3])
Res=[1,2,3];

Redo: insert(1,[2,3],[1,2,3])
Call: insert(1,[3],_x3149) <ENTER=call, ;=trace, h=help>: ;
 !!! clause: insert/3
Exit: insert(1,[3],[1,3])
Exit: insert(1,[2,3],[2,1,3])
Res=[2,1,3]q

?-
\end{verbatim}}

Starting with version 3.08 spy/1 and nospy/1 allow to watch
entry and exit from compiled predicates. Note that they
should be in
the file to be compiled, before any use
of the predicate to be spied on as in:

\begin{verbatim}
% FILE: jbond.pl
:-spy a/1.
:-spy c/1.

b(X):-a(X),c(X).

a(1).
a(2).

c(2).
c(3).
\end{verbatim}

{\flushleft This} gives the following interaction:


{\small \begin{verbatim}
?-[jbond].
......
?- b(X).

Call: a(_2158) <enter=call, other=trace>: ;
 !!! compiled(a/1)
Exit: a(1)
Call: c(1) <enter=call, other=trace>: ;
 !!! compiled(c/1)
Fail: c(1)
Redo: a(1)
Exit: a(2)
Call: c(2) <enter=call, other=trace>: ;
 !!! compiled(c/1)
Exit: c(2)
X=2;

Redo: c(2)
Fail: c(2)
Redo: a(2)
Fail: a(_2158)


no
\end{verbatim}}

{\flushleft Although} these are very basic debugging facilities you can
enhance them at your will and with some discipline in programming they
may be all you really need.  Anyway, future of debugging is definitely
not by tracing.  One thing is to have stronger static checking. In
dynamic debugging the way go is to have a database of trace-events and
then query it with high level tools.  We plan to add some non-tracing
database-oriented debugging facilities in the future.

\subsection{Debugging dynamically compiled code}

Let's suppose we load a file, let's say progs/cal.pl (a calendar
benchmark) in interpreted mode with:

\begin{verbatim}
?- ~cal.
\end{verbatim}

After running the benchmark (which, by the way gets such a big
speed-up through dynamic recompilation that it actually executes
faster than the benchmark witness empty loop) we would like to spy
on a predicate, with:

\begin{verbatim}
?-listing(day_of_week).
?-spy(day_of_week/4).
?-go(1). % toplevel goal
\end{verbatim}

Spy works as if no compilation has taken place! The trick is
that BinProlog can switch back to the interpreted version quite easily
as it has both in its working memory. You can at any time use
something like {\tt make\_static/1} or {\tt make\_dynamic/1} to explicitely chose
the desired execution format of a consulted predicate.
This is a major improvement on debugging programs, over previous
versions of BinProlog.

{\small \begin{verbatim}
?- go(1).
Call: day_of_week(1992,12,1,_x2474) <ENTER=call, ;=trace, h=help>: ;
 !!! clause: day_of_week/4
 Call: cal_key(12,_x2947,_x2948) <ENTER=call, ;=trace, h=help>: ;
  !!! clause: cal_key/3
 Exit: cal_key(12,4,0)
......................
......................
\end{verbatim}}

\section{Source-level stateless modules}

\begin{verbatim}
(module)/1
current_module/1
is_module/1          - checks/generates an existing module-name
module_call/2, ':'/2 - calls a predicate hidden in a module
module_name/3        - adds the name of a module to a symbol
module_predicate/3   - adds the name of a module to a goal
modules/1            - gives the list of existing modules
\end{verbatim}

The following example:
\begin{verbatim}
:-module m1.
:-public d/1.

a(1).
a(2).
a(3).
a(4).

d(X):-a(X).

:-module m2.

:-public b/1.

b(X):-c(X).

c(2).
c(3).
c(4).
c(5).
c(6).

:-module m3.

:-public test/1.

test(X):-b(X),d(X).

:-module user.

go:-modules(Ms),write(Ms),nl,fail.
go:-test(X),write(X),nl,fail.
\end{verbatim}

Executing goal `go' will generate the following output:

\begin{verbatim}
[user/0,m1/0,m2/0,m3/0]
2
3
4
\end{verbatim}

Starting with version 3.30, predicates in the BinProlog system
itself which are not intended to be used by applications, are hidden in
the module {\tt prolog} but can be accessed by calling them with
\begin{verbatim}
   'prolog:my_predicate'(...)
\end{verbatim}

Explicit naming of the module where the hidden predicate
is defined should be used when call/1, findall/3 etc. uses
a hidden predicate, even if it is in the module itself.

This draconian constraint is motivated by simplicity of
BinProlog's stateless purely source-level module system.
Basically predicates in a module have their names prefixed
as in
\begin{verbatim}
   my_current_module:my_predicate
\end{verbatim}
in a preprocessing
step, except if they are declared {\tt public} or are known to
the system as being so (i.e. in the case of builtins).

This basic concept of modules (essentially the same as
what can be achieved with {\tt extern} and {\tt static}
declarations in C) covers only compiled code,
and is mostly intended to ensure
multiple name spaces with a very simple semantics
and no additional space or time overhead. On the other
hand use of linear and intuitionistic implication
is suggested for dynamic modular and hypothetical reasoning
constructs.

{\em Meta-predicate} declarations are not supported at this time (mostly
because they are at least as cumbersome as just putting the right
name extension in argument positions which require it, but they
might be added in the future if a significant number of
users will ask to have them.

Note that builtins and predicates defined in a special module {\tt user}
are always public. A public predicate keeps its name unchanged
in the global name space while hidden predicates have their
names prefixed by the name of the module in their
definitions and in all their statically obvious (first-order)
uses.

Alternatively, {\tt module/2} allows to define a module and its
public predicates with one declaration as in:

\begin{verbatim}
:-module(beings,[cat/4,dog/4,chicken/2,fish/0,maple/1,octopus/255]).
\end{verbatim}

\section{Interoperation with Jinni}

BinProlog interoperates as a client or server with Jinni . Communication over
sockets is very fast, around 1000 exchanges per second.

The BinProlog side API is compatible with Jinni's server and client classes and supports socket reuse:

\begin{itemize}
\item {\tt rpc\_server(Port,Password)}: runs Jinni compatible server with socket reuse
\item {\tt rpc\_server}: runs Jinni compatible server with socket reuse on default port

\item {\tt rpc(Query)}: calls server on current local reusable socket]
\item {\tt rpc(Answer,Goal,Result)}: calls server on local reusable socket and
gets back Result as the(Answer) or no

\end{itemize}

Here is a Jinni client talking to a BinProlog server:

\begin{verbatim}
BinProlog Server Window:

?-rpc_server.

Jinni Client Window::

?-new(client,C),C:ask(println(hello)),C:disconnect.
hello
\end{verbatim}

Here is a BinProlog client talking to a Jinni server:

\begin{verbatim}
Jinni Server Window:

?- new(server,S),S:serve.
hello

BinProlog Client Window:

?- new_client(C),ask(C,println(hello)),stop_service(C).

\end{verbatim}

%\section{Mobile Code}

%The latest version of BinProlog supports two simple {\bf move/0} 
% and {\bf return/0}
%operations which transport
%computation to the server and back. 
%The client simply waits until computation completes, when bindings
% for the first solution are propagated back:

%{\small \begin{verbatim}
%Window 1: a mobile thread 

%?-there,move,println(on_server),member(X,[1,2,3]),
%        return,println(back). 
%back
%X=1; 
%no. 

%Window 2: a server 

%?-trust. 
%on_server 
%\end{verbatim}}

%In case return is absent, computation proceeds to the end of the transported continuation.
%Note that mobile computation is more expressive and more efficient than remote predicate calls as such. Basically, it {\em moves once}, and executes on the server {\em all future computations} of the current AND branch until a return instruction is hit,
%when it takes the remaining continuation and comes back. This can be seen by comparing real time execution speed for: 

%{\small \begin{verbatim}
%?-there,for(I,1,1000),run(println(I)),fail. 

%?-there,move,for(I,1,1000),println(I),fail. 
%\end{verbatim}}
 
%While the first query uses {\tt run/1} each time to send a remote task to the server,
%the second moves once the full computation to the server where it executes without
%further requiring network communications.
%Note that the {\tt move/0, return/0} pair cut nondeterminism for the transported segment
%of the current continuation. This avoids having to transport state of the choice-point stack
%as well as implementation complexity of multiple answer returns and tedious 
%distributed backtracking synchronization. Surprisingly, this is not a strong limitation,
%as the programmer can simply use something like:

%{\small \begin{verbatim}
%?-there,move,findall(X,for(I,1,1000),Xs),return,member(X,Xs).
%\end{verbatim}}

%\noindent to emulate (finite!) nondeterministic remote execution, by collecting all solutions
%at the remote side and exploring them through (much more efficient) 
%local backtracking after returning.


\section{Threads}

Windows and Linux versions of BinProlog also support multithreaded execution.
Type help(thread), then use info/1 for more information on thread related operations.

In fact, for most programs just using

{\small \begin{verbatim}
bg(Goal)

and

synchronize(Statement)
\end{verbatim}}

are all you need to get started.
Take a look at the StockMarket Web-based demo at
\begin{verbatim}
http://www.binnetcorp.com 
\end{verbatim}

for a more interesting example using multiple threads.
BinNet's {\em online demo} page contanins a naive reverse benchmark
allowing the user to mesure BinProlog's performance with various
thread/list lenght/iterations by filling in a Web form.

On a Pentium 4 2.4 GHz, depending on the parmeters, BinProlog
achieves between 120 and 130 million inferences/second (MegaLIPS).

\section{Calling BinProlog as a DLL}

The library bp\_lib.dll (available to users with BinProlog Source
or BinProlog Professional License) is called with parameters 
similar to BinProlog's command line. 

When a C string parameter parameter like:

\begin{verbatim}
  "call((consult(hello),go,halt))"
\end{verbatim}

\noindent is passed to the function {\bf bp\_main()} exported by the DLL,
BinProlog consults the file {\bf hello.pro}, starts with the goal
{\bf go} and finaly halts.

For more complex interaction with DLLs, we refer to BinProlog's C interface
described in \cite{bp7interface}.
Mor complex interaction patterns may require some adaptation of
the BinProlog sources, i.e. exporting some other functions and
calling back through new BinProlog builtins.
A Visual C/C++ project file is provided to simplify working with
the sources as well as command line makefiles (see directory BP\_DLL).

\section{Exceptions}
BinProlog support ISO Prolog exception handling.
Throwing and exception specified with throw/1 has
the result that the control leaves the current point
and the stack is searched until a matching catch/3
is found (a default one waits at toplevel for 
uncought exceptions). Catch(Goal,ExceptionPattern,Action),
executes Goal and if it catches an exception matching
ExceptionPattern, it executes Action, like in the
following example:

\begin{verbatim}
?- catch(
    (for(I,1,100),I=10,throw(boo(I)),println(too_late)),
    X,
    println(got(X))
  ).

got(boo(10))
I=_x2316,
X=boo(10)

yes
\end{verbatim}


\section{Using the Redistributable BinProlog Runtime Executable}
Starting with version 7.20, owners of BinProlog Professional
Edition can distribute their applications in binary format,
together with the BinProlog Runtime Executable (bpr.exe on
Windows, bor or bpr.<platform> on Unix).

To build a binary application do the following. Create
a file, for instance {\bf myfile.pl}, possibly including 
various modules in it with
\begin{verbatim}
:-[file1].
...
:-[fileN].
\end{verbatim}

Compile the myfile.pl file to a corresponding {\bf myfile.wam} file, 
using {\bf fcompile(myfile)} or {\bf scompile(myfile)} . 
The later contains a builtin {\em make} facility, 
for upgrading only the modules which have changed, 
while also loading the file into the development environment 
at the end.

To execute the resulting {\bf myfile.wam} binary application,
type:

\begin{verbatim}
bpr <myfile>.wam
\end{verbatim}

The BinProlog Runtime Executable (bpr.exe or bpr) has all
the abilities of BinProlog, except the ability to compile
applications to files or memory. However, it is able to
consult and interpret Prolog data files, by asserting their
content into the dynamic database. Executing complex code
with the interpreter is, however, much slower than compiling,
pleas make sure that the code you deliver has been previously\
compiled to the *.wam form, which is executed by the runtime
system at full speed.


%\pagebreak
\section{Example programs}
The directory {\tt progs} contains a few BinProlog benchmarks and applications.

{\small
\begin{verbatim}
allperms.pl: permutation benchmarks with findall
bestof.pl:   an implementation  of bestof/3

bfmeta.pl:   breadth-first metainterpreter
backprop.pl: float intensive neural net learning by back-propagation
cal.pl:      calendar: computes the last 10000 fools-days
chat.pl:     CHAT parser
choice.pl:   Choice-intensive ECRC benchmark
cbrev.pl:    nrev with ^/2 as a constructor instead of ./2
cube.pl:     E. Tick's benchmark program
fibo.pl:     naive Fibonacci
ffibo.pl:    naive Fibonacci with floats
mfibo.pl:    Fibonacci with memoing
dfibo.pl:    Fibonacci with Delphi memoing
hello.pl:    example program to create stand-alone Unix application
knight.pl:   knight tour to cover a chess-board (uses the bboard)
lknight.pl:  knight tour to cover a chess-board (uses the lists)
ltak.pl:     tak program with lemmas
lfibo.pl:    fibo program with lemmas
lq8.pl :     8 queens using global logical variables
maplist.pl:  fast findall based maplist predicate
brev.pl:     naive reverse benchmark
nrev30.pl:   small nrev benchmark to reconsult for the meta-interpreter
or.pl:       or-parallel simulator for binary programs (VT100)
other_bm*:   benchmark suite to compare Sicstus, Quintus and BinProlog
puzzle.pl:   king-prince-queen puzzle
q8.pl:       fast N-queens
qrev.pl:     quick nrev using builtin det_append/3
subset.pl:   findall+subset
tetris.pl:   tetris player (VT100)
\end{verbatim}
}


\section{Appendix}

\subsection{Default Operator Definition}

Note the operators when used as non-operator atoms {\em need to be in paranthesises}, i.e. you should say write([this,(module), is, (not), in, the,(public),domain]) instead of write([this,module, is, not, in, the,public,domain]).

BinProlog's default operator definitions
(see file oper.pl) are the following:

{\small
\begin{verbatim}
:-op(1000,xfy,',').
:-op(1100,xfy,(';')).

:-op(1200,xfx,('-->')).
:-op(1200,xfx,(':-')).
:-op(1200,fx,(':-')).
:-op(700,xfx,'is').
:-op(700,xfx,'=').

:-op(1050,xfx,(@@)).

:-op(500,yfx,'-').
:-op(200,fy,'-').

:-op(500,yfx,'+').
:-op(200,fy,'+').

:-op(400,yfx,'/').
:-op(400,yfx,'*').
:-op(400,fx,'*').
:-op(400,yfx,(mod)).
:-op(200,yfx,(**)).
:-op(200,xfy,(^)).

:-op(300,fy,(~)).
:-op(650,xfy,'.').
:-op(660,xfy,'++').

:-op(700,xfx,'>=').
:-op(700,xfx,'>').
:-op(700,xfx,'=<').
:-op(700,xfx,(<)).
:-op(700,xfx,(=\=)).
:-op(700,xfx,(=:=)).

:-op(400,yfx,(>>)).
:-op(400,yfx,(<<)).
:-op(400,yfx,(//)).

:-op(200,yfx,(\/)).
:-op(200,yfx,(/\)).
:-op(200,yfx,(\)).
:-op(200,fx,(\)).

:-op(700,xfx,(@>=)).
:-op(700,xfx,(@=<)).
:-op(700,xfx,(@>)).
:-op(700,xfx,(@<)).

:-op(700,xfx,(\==)).
:-op(700,xfx,(==)).
:-op(700,xfx,(=..)).
:-op(700,xfx,(\=)).

:-op(900,fy,(not)).
:-op(900,fy,(\+)).
:-op(900,fx,(spy)).
:-op(900,fx,(nospy)).

:-op(950,fx,(##)).

:-op(950,xfy,(=>)).
:-op(950,xfx,(<=)).

:-op(1050,xfy,(->)).

:-op(1150,fx,(dynamic)).
:-op(1150,fx,(public)).
:-op(1150,fx,(module)).
:-op(1150,fx,(mode)).

:-op(1150,fx,(multifile)).
:-op(1150,fx,(discontiguous)).

:-op(1200,xfx,(::-)).

:-op(50,yfx,(:)).

:-op(100,fx,(@)).

:-op(50,fx,(^)).

:-op(500,fx,(#>)).
:-op(500,fx,(#<)).
:-op(500,fx,(#:)).
:-op(500,fx,(#+)).
:-op(500,fx,(#*)).
:-op(500,fx,(#=)).
:-op(500,fx,(#-)).
:-op(500,fx,(#?)).
\end{verbatim}
}

\section{Further readings}

Related BinProlog documentation is available at: 
\cite{bp7user,bp7advanced,bp7interface,bp7crossref}.

\bibliography{tarau}

\end{document}

